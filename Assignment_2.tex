\documentclass[11pt,fleqn]{article}

\setlength {\topmargin} {-.15in}
\setlength {\textheight} {8.6in}

\usepackage{amsmath}
\usepackage{amssymb}
\usepackage{amsthm}
\usepackage{color}

\renewcommand{\labelenumi}{\theenumi.}
\renewcommand{\labelenumii}{\theenumii.}
\renewcommand{\labelenumiii}{\theenumiii.}
\newcommand{\be}{\begin{enumerate}}
\newcommand{\ee}{\end{enumerate}}
\newcommand{\bi}{\begin{itemize}}
\newcommand{\ei}{\end{itemize}}
\newcommand{\bc}{\begin{center}}
\newcommand{\ec}{\end{center}}
\newcommand{\bsp}{\begin{sloppypar}}
\newcommand{\esp}{\end{sloppypar}}
\newcommand{\mname}[1]{\mbox{\sf #1}}
\newcommand{\pnote}[1]{{\langle \text{#1} \rangle}}

\begin{document}

\begin{center}

  {\large \textbf{COMPSCI/SFWRENG 2FA3}}\\[2mm]
  {\large \textbf{Discrete Mathematics with Applications II}}\\[2mm]
  {\large \textbf{Winter 2021}}\\[8mm]
  {\huge \textbf{Assignment 2}}\\[6mm]
  {\large \textbf{Dr.~William M. Farmer and Dr.~Mehrnoosh Askarpour}}\\[2mm]
  {\large \textbf{McMaster University}}\\[6mm]
  {\large Revised: February 2, 2021}

\end{center}

\medskip

Assignment 2 consists of two problems.  You must write your solutions
to the problems using LaTeX.

Please submit Assignment~2 as two files,
\texttt{Assignment\_2\_\emph{YourMacID}.tex} and
\texttt{Assignment\_2\_\emph{YourMacID}.pdf}, to the Assignment~2
folder on Avenue under Assessments/Assignments.
\texttt{\emph{YourMacID}} must be your personal MacID (written without
capitalization).  The \texttt{Assignment\_2\_\emph{YourMacID}.tex}
file is a copy of the LaTeX source file for this assignment
(\texttt{Assignment\_2.tex} found on Avenue under
Contents/Assignments) with your solution entered after each problem.
The \texttt{Assignment\_2\_\emph{YourMacID}.pdf} is the PDF output
produced by executing

\begin{itemize}

  \item[] \texttt{pdflatex Assignment\_2\_\emph{YourMacID}}

\end{itemize}

This assignment is due \textbf{Sunday, February 7, 2020 before
  midnight.}  You are allow to submit the assignment multiple times,
but only the last submission will be marked.  \textbf{Late submissions
  and files that are not named exactly as specified above will not be
  accepted!}  It is suggested that you submit your preliminary
\texttt{Assignment\_2\_\emph{YourMacID}.tex} and
\texttt{Assignment\_2\_\emph{YourMacID}.pdf} files well before the
deadline so that your mark is not zero if, e.g., your computer fails
at 11:50 PM on February 7.

\textbf{Although you are allowed to receive help from the
  instructional staff and other students, your submission must be your
  own work.  Copying will be treated as academic dishonesty! If any of
  the ideas used in your submission were obtained from other students
  or sources outside of the lectures and tutorials, you must
  acknowledge where or from whom these ideas were obtained.}

\newpage

\subsection*{Problems}

\be

  \item \textbf{[10 points]}

Let \mname{SimpleTree} be the inductive
    set defined by the following constructors:

  \be

    \item $\mname{Leaf} : \mathbb{N} \rightarrow \mname{SimpleTree}$.

    \item $\mname{Branch1} : \mname{SimpleTree} \rightarrow
      \mname{SimpleTree}$.

    \item $\mname{Branch2} : \mname{SimpleTree} \times
      \mname{SimpleTree} \rightarrow \mname{SimpleTree}$.

  \ee

  The function $\mname{leaves} : \mname{SimpleTree} \rightarrow
  \mathbb{N}$ is defined by recursion and pattern matching as:

  \be

    \item $\mname{leaves}(\mname{Leaf}(n)) = 1$.

    \item $\mname{leaves}(\mname{Branch1}(t)) = \mname{leaves}(t)$.

    \item $\mname{leaves}(\mname{Branch2}(t_1,t_2)) =
      \mname{leaves}(t_1) + \mname{leaves}(t_2)$.

  \ee

  The function $\mname{branches} : \mname{SimpleTree} \rightarrow
  \mathbb{N}$ is defined by recursion and pattern matching as:

  \be

    \item $\mname{branches}(\mname{Leaf}(n)) = 0$.

    \item $\mname{branches}(\mname{Branch1}(t)) = 1 + \mname{branches}(t)$.

    \item $\mname{branches}(\mname{Branch2}(t_1,t_2)) = 1 +
      \mname{branches}(t_1) + \mname{branches}(t_2)$.

  \ee

  Prove that, for all $t \in \mname{SimpleTree}$, \[\mname{leaves}(t) \le
  \mname{branches}(t) + 1.\]

  \textcolor{blue}{\textbf{Qiang Gao, gaoq20, 2/6/2021}}

\begin{proof}
Let $P(t) \equiv \mname{leaves}(t) \le
  \mname{branches}(t) + 1$.  We will prove $P(t)$
for all $t \in \mname{SimpleTree}$ by strong induction.

\medskip

\emph{Base case}: $t = \mname{Leaf}(n)$.  We must show $P(\mname{Leaf}(n))$.
\begin{align*}
  &$P(\mname{Leaf}(n))$\\
  &\equiv \mname{leaves}(\mname{Leaf}(n)) \le \mname{branches}(\mname{Leaf}(n)) + 1 \\
  &\equiv 1 \le \mname{branches}(\mname{Leaf}(n)) + 1          & \pnote{definition of $\mname{leaves}$}\\
  &\equiv 1 \le 0 + 1          & \pnote{definition of $\mname{branches}$}\\
  &\equiv 1 \le 1 	& \pnote{arithmetic}\\
  &\equiv True
\end{align*}
So $P(\mname{Leaf}(n))$ holds.

\medskip

\emph{Induction step}:
\emph{Case 1}: We assume $P(t)$ holds for $t \in \mname{SimpleTree}$, prove $P(\mname{Branch1}(t))$
\begin{align*}
  &\mname{leaves}(\mname{Branch1}(t))\\
  &= \mname{leaves}(t) & \pnote{definition of $\mname{leaves}$}\\
  &\le \mname{branches}(t) + 1  & \pnote{induction hypothesis}\\
  &\le \mname{branches}(t) + 1 + 1 & \pnote{arithmetic}\\
  &= \mname{branches}(\mname{Branch1}(t)) + 1 & \pnote{definition of $\mname{branches}$}\\
\end{align*}
So $P(\mname{Branch1}(t))$ holds.

\medskip

\emph{Case 2}: We assume $P(t_1)$ and $P(t_2)$ holds for $t_1,$t_2 \in \mname{SimpleTree}$, prove $P(\mname{Branch2}(t_1,t_2))$
\begin{align*}
  &\mname{leaves}(\mname{Branch2}(t_1,t_2))\\
  &= \mname{leaves}(t_1) + \mname{leaves}(t_2) & \pnote{definition of $\mname{leaves}$}\\
  &\le \mname{branches}(t_1) + 1 + \mname{branches}(t_2) + 1 & \pnote{induction hypothesis}\\
  &= \mname{branches}(\mname{Branch2}(t_1,t_2)) + 1 & \pnote{definition of $\mname{branches}$}\\
\end{align*}
So $P(\mname{Branch2}(t_1,t_2))$ holds.

\medskip

Therefore, $P(t)$ holds for all $t \in \mname{SimpleTree}$ by strong induction.
\end{proof}

  \bigskip

  \item \textbf{[10 points]} 

  Let \mname{BinNum} be the inductive set defined by the following
  constructors:

  \bi

    \item[] $\mname{Zero} : \mname{BinNum}$.

    \item[] $\mname{One} : \mname{BinNum}$.

    \item[] $\mname{JoinZero} : \mname{BinNum} \rightarrow \mname{BinNum}$.

    \item[] $\mname{JoinOne} : \mname{BinNum} \rightarrow \mname{BinNum}$.

  \ei

  The members of \mname{BinNum} represent binary numerals like $1011$
  and $010$. \mname{Zero} represents 0; \mname{One} represents 1; and
  if $u$ represents U, then $\mname{JoinZero}(u)$ represents U$0$
  and $\mname{JoinOne}(u)$ represents U$1$.  For
  example, \[\mname{JoinOne}(\mname{JoinZero}(\mname{JoinOne}(\mname{One})))\]
  represents the binary number 1101.

  The function \[\mname{len} : \mname{BinNum} \rightarrow \mathbb{N}\]
  maps a member of \mname{BinNum} to its length.  \mname{len} is
  defined by the following equations using recursion and pattern
  matching:

  \bi

    \item[] $\mname{len}(\mname{Zero}) = 1$.

    \item[] $\mname{len}(\mname{One}) = 1$.

    \item[] $\mname{len}(\mname{JoinZero}(u)) = \mname{len}(u) + 1$.

    \item[] $\mname{len}(\mname{JoinOne}(u)) = \mname{len}(u) + 1$.

  \ei

  The function \[\mname{val} : \mname{BinNum} \rightarrow \mathbb{N}\]
  maps a member of \mname{BinNum} to the value of the binary numeral
  it represents.  For
  example, \[\mname{val}(\mname{JoinOne}(\mname{JoinZero}(\mname{JoinOne}(\mname{One}))))
  = (1101)_2 = 13.\] \mname{val} is defined by the following equations
  using recursion and pattern matching:

  \bi

    \item[] $\mname{val}(\mname{Zero}) = 0$.

    \item[] $\mname{val}(\mname{One}) = 1$.

    \item[] $\mname{val}(\mname{JoinZero}(u) = 2 * \mname{val}(u)$.

    \item[] $\mname{val}(\mname{JoinOne}(u) = (2 * \mname{val}(u)) +  1$.

  \ei

  The function \[\mname{add} : \mname{BinNum} \times \mname{BinNum}
  \rightarrow \mname{BinNum}\] is intended to implement addition on
  members of \mname{BinNum}.  It is defined by the following equations
  using recursion and pattern matching:

  \bi

    \item[] $\mname{add}(u,\mname{Zero}) = u$.

    \item[] $\mname{add}(\mname{Zero},u) = u$.

    \item[] $\mname{add}(\mname{One},\mname{One}) =
      \mname{JoinZero}(\mname{One})$.

    \item[] $\mname{add}(\mname{JoinZero}(u),\mname{One}) =
      \mname{JoinOne}(u)$.

    \item[] $\mname{add}(\mname{One},\mname{JoinZero}(u)) =
      \mname{JoinOne}(u)$.

    \item[] $\mname{add}(\mname{JoinOne}(u),\mname{One}) =
      \mname{JoinZero}(\mname{add} (u,\mname{One})$.

    \item[] $\mname{add}(\mname{One},\mname{JoinOne}(u)) =
      \mname{JoinZero}(\mname{add} (u,\mname{One})$.

    \item[] $\mname{add}(\mname{JoinZero}(u),\mname{JoinZero}(v)) =
      \mname{JoinZero}(\mname{add} (u,v)$.

    \item[] $\mname{add}(\mname{JoinOne}(u),\mname{JoinZero}(v)) =
      \mname{JoinOne}(\mname{add} (u,v)$.

    \item[] $\mname{add}(\mname{JoinZero}(u),\mname{JoinOne}(v)) =
      \mname{JoinOne}(\mname{add} (u,v)$.

    \item[] $\mname{add}(\mname{JoinOne}(u),\mname{JoinOne}(v)) =
      \mname{JoinZero}(\mname{add} (\mname{add} (u,v),\mname{One}))$.

  \ei

  Notice that the algorithm behind the definition is essentially the
  same algorithm that children learn to add numbers represented as
  decimal numerals.  The last equation is a bit complicated because it
  involves a carry of 1.

  \textbf{Lemma 1}. For all $u,v \in \mname{BinNum}$,
  \[\mname{len}(\mname{add}(u,v)) \le \mname{len}(u) +
  \mname{len}(v).\]

  \textbf{Theorem 1}. For all $u,v \in
  \mname{BinNum}$, \[\mname{val}(\mname{add}(u,v)) = \mname{val}(u) +
  \mname{val}(v).\]

  Theorem 1 states that \mname{add} correctly implements addition on
  the members of $\mname{BinNum}$.

  Prove Theorem 1 assuming Lemma 1.  (You are not required to prove
  Lemma 1.)  Hint: Use strong induction with $P(n) \equiv
  \mname{val}(\mname{add}(u,v)) = \mname{val}(u) + \mname{val}(v)$ for
  all $u,v \in \mname{BinNum}$ such that $n = \mname{len}(u) +
  \mname{len}(v)$.

  \bigskip

  \textcolor{blue}{\textbf{Qiang Gao, gaoq20, 2/6/2021}}
\\\\
In order to prove Theorem 1 better, we introduce Lemma 2 and Lemma 3.
\textbf{Lemma 2}. For all $u,v \in \mname{BinNum}$

$\mname{len}(u) + \mname{len}(v) \le \mname{len}(\mname{JoinOne}(u)) + \mname{len}(\mname{JoinOne}(v))$

\textbf{Lemma 3}. For all $u,v \in \mname{BinNum}$

$\mname{len}(\mname{add}(u,v)) +\mname{len}(\mname{JoinOne}(\mname{Zero})) \le \mname{len}(\mname{JoinOne}(u)) + \mname{len}(\mname{JoinOne}(v))$
  
\begin{proof}:
\\\\
\emph{Lemma 2}:
\begin{align*}
  & \mname{len}(u) + \mname{len}(v)\\
  & \le \mname{len}(u) + 1 + \mname{len}(v) + 1 & \pnote{arithmetic}\\
  &= \mname{len}(\mname{JoinOne}(u)) + \mname{len}(\mname{JoinOne}(v)) & \pnote{definition of \mname{len}}\\
\end{align*}


\emph{Lemma 3}:
\begin{align*}
  & \mname{len}(\mname{add}(u,v)) + \mname{len}(\mname{JoinOne}(\mname{Zero}))  & \pnote{LHS}\\
  & \le \mname{len}(u) + \mname{len}(v) + \mname{len}(\mname{JoinOne}(\mname{Zero}))  & \pnote{Lemma 1}\\
  &= \mname{len}(u) + \mname{len}(v) +  \mname{len}(\mname{Zero}) + 1  & \pnote{definition of \mname{len}}\\
  &= \mname{len}(u) + \mname{len}(v) +1 + 1  & \pnote{definition of \mname{len}}\\
  &= \mname{len}(\mname{JoinOne}(u)) + \mname{len}(\mname{JoinOne}(v))  & \pnote{definition of \mname{len}}\\
\end{align*}

We now prove \textbf{Thereom 1}

\emph{Proof}: We assume $P(a)$ for all a $<$ n, and prove $P(n)$ by strong induction.

\emph{Case 1}: $v = \mname{Zero}$
\begin{align*}
  &\mname{val}(\mname{add}(u,\mname{Zero})) & \pnote{LHS}\\
  &= \mname{val}(u)  & \pnote{definition of \mname{add}}\\
  &= \mname{val}(u) + 0 & \pnote{arithmetic}\\
  &= \mname{val}(u) +  \mname{val}(\mname{Zero}) & \pnote{definition of \mname{val}}\\
\end{align*}


\emph{Case 2}: $u = \mname{Zero}$.
\begin{align*}
  &\mname{val}(\mname{add}(\mname{Zero},v)) & \pnote{LHS}\\
  &= \mname{val}(u)  & \pnote{definition of \mname{add}}\\
  &= 0 + \mname{val}(u)  & \pnote{arithmetic}\\
  &= \mname{val}(\mname{Zero}) +  \mname{val}(v) & \pnote{definition of \mname{val}}\\
\end{align*}


\emph{Case 3}: $u = \mname{JoinZero}(u\prime)$ and $v = \mname{JoinZero}(v\prime)$
\begin{align*}
  &\mname{val}(\mname{add}(\mname{JoinZero}(u\prime),\mname{JoinZero}(v\prime))) & \pnote{LHS}\\
  &= \mname{val}(\mname{JoinZero}(\mname{add}(u\prime,v\prime)))  & \pnote{definition of \mname{add}}\\
  &= 2 * \mname{val}((\mname{add}(u\prime,v\prime)))  & \pnote{definition of \mname{val}}\\
  &= 2 * ( \mname{val}(u\prime)+\mname{val}(v\prime)) & \pnote{Induction Hypothesis}\\
  &= 2 * ( \mname{val}(u\prime)) + 2 * ( \mname{val}(v\prime)) & \pnote{arithmetic}\\
  &=  \mname{val}(\mname{JoinZero}(u\prime)) + \mname{val}(\mname{JoinZero}(v\prime)) 
  & \pnote{definition of \mname{val}}\\
\end{align*}

\emph{Case 4}: $u =\mname{JoinOne}(u\prime)$, $v =\mname{JoinZero}(v\prime)$
\begin{align*}
  &\mname{val}(\mname{add}(\mname{JoinOne}(u\prime),\mname{JoinZero}(v\prime))) & \pnote{LHS}\\
  &= \mname{val}(\mname{JoinOne}(\mname{add}(u\prime,v\prime)))  & \pnote{definition of \mname{add}}\\
  &= 2 * \mname{val}((\mname{add}(u\prime,v\prime))) + 1  & \pnote{definition of \mname{val}}\\
  &= 2 * ( \mname{val}(u\prime)+\mname{val}(v\prime)) + 1 & \pnote{Induction Hypothesis}\\
  &= 2 * ( \mname{val}(u\prime)) + 1 + 2 * ( \mname{val}(v\prime)) & \pnote{arithmetic}\\
  &=  \mname{val}(\mname{JoinOne}(u\prime)) + \mname{val}(\mname{JoinZero}(v\prime)) 
  & \pnote{definition of \mname{val}}\\
\end{align*}

\emph{Case 5}: $u =\mname{JoinZero}(u\prime)$, $v =\mname{JoinOne}(v\prime)$. 
\begin{align*}
  &\mname{val}(\mname{add}(\mname{JoinZero}(u\prime),\mname{JoinOne}(v\prime))) & \pnote{LHS}\\
  &= \mname{val}(\mname{JoinOne}(\mname{add}(u\prime,v\prime)))  & \pnote{definition of \mname{add}}\\
  &= 2 * \mname{val}((\mname{add}(u\prime,v\prime))) + 1  & \pnote{definition of \mname{val}}\\
  &= 2 * ( \mname{val}(u\prime)+\mname{val}(v\prime)) + 1 & \pnote{Induction Hypothesis}\\
  &= 2 * ( \mname{val}(u\prime)) + 2 * ( \mname{val}(v\prime)) + 1 & \pnote{arithmetic}\\
  &=  \mname{val}(\mname{JoinZero}(u\prime)) + \mname{val}(\mname{JoinOne}(v\prime)) 
  & \pnote{definition of \mname{val}}\\
\end{align*}


\emph{Case 6}:  $u =\mname{JoinOne}(u\prime)$, $v =\mname{JoinOne}(v\prime)$
\begin{align*}
  &\mname{val}(\mname{add}(\mname{JoinOne}(u\prime),\mname{JoinOne}(v\prime))) 
  & \pnote{LHS}\\
  &= \mname{val}((\mname{JoinZero}(\mname{add}(\mname{add}(u\prime, v\prime), \mname{One})))
  &= 2 * \mname{val}(\mname{add}(\mname{add}(u\prime, v\prime), \mname{One}))
  & \pnote{definition of \mname{add}}\\
  &= 2 * (\mname{val}(\mname{add}(u\prime,v\prime)) + \mname{val}(\mname{One}))
  & \pnote{definition of \mname{val}}\\
  &= 2 * (\mname{val}(u\prime) + \mname{val}(v\prime + \mname{val}(\mname{One}))
  & \pnote{Induction Hypothesis}\\
  &= 2 * (\mname{val}(u\prime) + \mname{val}(v\prime + 1)
  & \pnote{definition of \mname{val}}\\
  &= (2 * \mname{val}(u\prime) + 1) + (2 * \mname{val}(v\prime) + 1) 
  & \pnote{arithmetic}\\
  &= \mname{val}(\mname{JoinOne}(u\prime)) + \mname{val}(\mname{JoinOne}(v\prime)) 
  & \pnote{definition of \mname{val}}\\
\end{align*}

So P(n) has been successfully proved by strong induction.
\end{proof}

\ee

\end{document}


